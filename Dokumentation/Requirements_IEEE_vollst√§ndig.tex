\documentclass[a4paper]{article}

\usepackage[utf8]{inputenc}

\title{Requirements}

\begin{document}
	
	\maketitle
	
\textbf{R1.)} KorrekteArchitekturDerSmartphones \\
\textbf{Funktion:} Die App sollte auf Smartphones mit Android 5.1 oder höher, Internet-Zugang und einer armeabi, armeabi-v7a oder amr64-v8a Architektur lauffähig sein. \\
\textbf{Beschreibung:} Die App sollte auf Smartphones mit Android 5.1 oder höher, Internet-Zugang und einer armeabi, armeabi-v7a oder amr64-v8a Architektur lauffähig sein. \\
\textbf{Quelle:} Treffen am 30.04.2018 \\
\textbf{Abhängigkeiten:} - \\\\


\textbf{R2.)} VerhaltenOhneInternetverbindung \\
\textbf{Funktion:} Die App sollte bei fehlender Internetverbindung mit eingeschränkter Funktionalität weiter nutzbar sein. \\
\textbf{Beschreibung:} Bricht die Verbindung zum Internet ab, sollte die App robust darauf reagieren. Einige Grundfunktionen sollten dabei weiter nutzbar sein. \\
\textbf{Quelle:} Treffen am 30.04.2018 \\
\textbf{Abhängigkeiten:} - \\\\


\textbf{R3.)} VisuelleDarstellungDerVorhersage \\
\textbf{Funktion:} Die Vorhersagen sollen visuell dargestellt werden. \\
\textbf{Beschreibung:} Die Berechnungen zur Vorhersage der wahrscheinlichen Vogelposition sollen visuell in der App angezeigt werden. \\
\textbf{Quelle:} Treffen am 26.04.2018 \\
\textbf{Abhängigkeiten:} - \\\\


\textbf{R4.)} DatenLadenAusDerMovebank \\
\textbf{Funktion:} Daten sollen aus der Movebank geladen werden. \\
\textbf{Beschreibung:} Das Laden der für die Berechnung benötigten Daten geschieht ausschließlich über die Movebank, die die gegeben Daten zur Verfügung stellt. \\
\textbf{Quelle:} Treffen am 26.04.2018 \\
\textbf{Abhängigkeiten:} - \\\\


\textbf{R5.)} VisualisierungDerVorhandenenDaten \\
\textbf{Funktion:} Die vorhandenen Daten für einen gegebenen Vogel und Zeitraum sollen visualisiert werden. \\
\textbf{Beschreibung:} Wird ein Vogel ausgewählt, sollen dessen vorhandenen Daten in einem bestimmtem Zeitraum angezeigt werden können.\\
\textbf{Quelle:} Treffen am 26.04.2018 \\
\textbf{Abhängigkeiten:} R4 \\\\


\textbf{R6.)} BerechnungDesAufenthaltsraumes \\
\textbf{Funktion:} Für einen gegebenen Vogel soll ein wahrscheinlicher Aufenthaltsraum berechnet werden. \\
\textbf{Beschreibung:} Wird ein Vogel ausgewählt, soll aufgrund seiner vergangenen Daten der Aufenthaltsraum berechnet werden, in der er sich am wahrscheinlichsten befindet. \\
\textbf{Quelle:} Treffen am 26.04.2018 \\
\textbf{Abhängigkeiten:} R4 \\\\



\textbf{R7.)} AufteilungDerPositionsvorhersage \\
\textbf{Funktion:} Die Positionsvorhersage soll in Zonen mit unterschiedlicher Wahrscheinlichkeit unterteilt werden. \\
\textbf{Beschreibung:} Um die Vorhersage übersichtlichter und verständlicher zu machen, soll die Positionsvorhersagen in Zonen eingeteilt werden.  \\
\textbf{Quelle:} Treffen am 30.04.2018 \\
\textbf{Abhängigkeiten:} R6 \\\\


\textbf{R8.)} VorhersageFürBestimmtenZeitraum \\
\textbf{Funktion:} Die Positionsvorhersage soll für einen gegebenen Zeitraum berechnet werden. \\
\textbf{Beschreibung:} Die Positionsvorhersage soll für einen gegeben Zeitraum ab dem aktuellen Moment berechnet werden \\
\textbf{Quelle:} Treffen am 26.04.2018 \\
\textbf{Abhängigkeiten:} R4 \\\\





\textbf{R9.)} ModularitätDerSoftware\\
\textbf{Funktion:} Die Software soll modular gestaltet sein\\
\textbf{Beschreibung:} Die Software soll aus Modulen aufgebaut werden. Diese Module sollen möglichst unabhängig voneinander sein.\\
\textbf{Quelle:} Treffen am 30.04.2018\\
\textbf{Abhängigkeiten:} -\\\\

\textbf{R10.)} EinheitlicheSchnittstelleFürVorhersagealgorithmen\\
\textbf{Funktion:} Die Software soll über eine einheitliche Schnittstelle für Vorhersagealgorithmen verfügen\\
\textbf{Beschreibung:} Zum einfachen Hinzufügen von weiteren Vorhersagealgorithmen soll die Software über ein einheitliches Interface für diese verfügen\\
\textbf{Quelle:} Treffen am 26.04.2018\\
\textbf{Abhängigkeiten:} R9\\\\

\textbf{R11.)} SucheNachVogel\\
\textbf{Funktion:} In der Software soll nach einem Vogel gesucht werden können\\
\textbf{Beschreibung:} Der Benutzer soll anhand einer ID nach einem bestimmten Vogel suchen können\\
\textbf{Quelle:} Treffen am 26.04.2018\\
\textbf{Abhängigkeiten:} -\\\\

\textbf{R12.)} VisualisierungInCesiumBeiWLAN\\
\textbf{Funktion:} Bei WLAN-Zugang sollen die Daten in Cesium visualisiert werden\\
\textbf{Beschreibung:} Wenn das Smartphone Internetzugang über WLAN hat sollen die getrackten Positionen und und die Vorhersagen mit Cesium visualisiert werden\\
\textbf{Quelle:} Treffen am 26.04.2018\\
\textbf{Abhängigkeiten:} R4, R5, R6, R7, R8\\\\

\textbf{R13.)} VisualisierungMitMapbox\\
\textbf{Funktion:} Bei Zugang zum mobilen Internet sollen die Daten mit Mapbox visualisiert werdne\\
\textbf{Beschreibung:} Wenn das Smartphone Internetzugang ausschließlich über mobieles Internet hat sollen die getrackten Positionen und und die Vorhersagen mit Mapbox visualisiert werden\\
\textbf{Quelle:} Treffen am 30.04.2018\\
\textbf{Abhängigkeiten:} R4, R5, R6, R7, R8\\\\

\textbf{R14.)} VisualisierungAufOfflineKarten\\
\textbf{Funktion:} Ohne Internetzugang sollen die Daten auf offline-Karten visualisiert werden\\
\textbf{Beschreibung:} Falls offline-Karten vorhanden sind sollen die getrackten Positionen und und die Vorhersage auf diesen visualisiert werden.
\textbf{Quelle:} Treffen am 30.04.2018\\
\textbf{Abhängigkeiten:} R4, R5, R6, R7, R8\\\\

\textbf{R15.)} WählbarerVorhersagezeitraum\\
\textbf{Funktion:} Der Nutzer soll den Zeitraum für die Vorhersage wählen können\\
\textbf{Beschreibung:} Der Nutzer soll die Möglichkeit haben einzustellen für welchen Zeitraum die Vorhersage berechnet wird.
\textbf{Quelle:} Treffen am 26.04.2018\\
\textbf{Abhängigkeiten:} - \\\\

\textbf{R16.)} WahlDerDatenFürVorhersage\\
\textbf{Funktion:} Der Nutzer soll einstellen können auf welchen Daten die Vorhersage basiert\\
\textbf{Beschreibung:} Der Nutzer soll einstellen können auf welchen der verfügbaren Daten und Parameter die Vorhersage basiert.\\
\textbf{Quelle:} Treffen am 26.04.2018\\
\textbf{Abhängigkeiten:} R4 \\\\




\textbf{R17.)} RobustGegenüberFormalFalschenDaten \\
\textbf{Funktion:} Die App soll robust gegenüber formal falschen Daten aus der Movebank sein \\
\textbf{Beschreibung:} Die App soll formal falsche Daten, die sie aus der Movebank bekommt, so aufbereiten, dass das Berechnen und Visualisieren eines Ergebnisses möglich ist. \\
\textbf{Quelle:} Treffen am 26.04.2018 \\
\textbf{Abhängigkeiten:} R4 \\\\

\textbf{R18.)} HinweisAufFormalFalscheDaten \\
\textbf{Funktion:} Der Nutzer soll auf einen hohen Anteil formal falscher Daten hingewiesen werden \\
\textbf{Beschreibung:} Die App soll dem Nutzer einen Hinweis anzeigen, wenn viele formal falsche Daten zur Berechnung verwendet werden mussten, damit dieser einordnen kann wie aussagekräftig das Ergebnis ist. \\
\textbf{Quelle:} Treffen am 30.04.2018 \\
\textbf{Abhängigkeiten:} R4 \\\\

\textbf{R19.)} BerechnungszeitraumVorhersage \\
\textbf{Funktion:} Der Berechnungszeitraum der Vorhersage soll den vom Nutzer erwarteten Zeitraum nicht deutlich überschreiten \\
\textbf{Beschreibung:} Die App soll die Vorhersage in einem adäquaten Zeitraum berechnen und diese dem Nutzer möglichst schnell visualisieren. \\
\textbf{Quelle:} Treffen am 30.04.2018 \\
\textbf{Abhängigkeiten:} R3, R4, R5, R6, R7, R8, R16, R17 \\\\

\textbf{R20.)} TouchBasierteBenutzeroberfläche \\
\textbf{Funktion:} Die App soll über eine touch-basierte Benutzeroberfläche verfügen \\
\textbf{Beschreibung:} Die App soll über eine touch-basierte Benutzeroberfläche bedienbar sein. \\
\textbf{Quelle:} Treffen am 26.04.2018 \\
\textbf{Abhängigkeiten:} - \\\\

\textbf{R21.)} OfflineDatenSpeicherung \\
\textbf{Funktion:} Die App soll für die aktuelle Vorhersage Offline-Daten speichern \\
\textbf{Beschreibung:} Die App soll die aktuelle Vorhersage lokal speichern, damit man diese auch ohne aktiven Internetzugang betrachten kann. \\
\textbf{Quelle:} Treffen am 26.04.2018 \\
\textbf{Abhängigkeiten:} R6 \\\\

\textbf{R22.)} AktualisierungÜberMobileDaten \\
\textbf{Funktion:} Über mobile Daten soll die aktuelle Vorhersage aktualisiert werden können \\
\textbf{Beschreibung:} Die App soll auch über mobile Daten aktuelle Daten aus der Movebank empfangen und für eine aktuelle Vorhersage verwenden können. \\
\textbf{Quelle:} Treffen am 30.04.2018 \\
\textbf{Abhängigkeiten:} R3, R4, R5, R6, R7, R8, R13 \\\\

\textbf{R23.)} CodeDokumentation \\
\textbf{Funktion:} Programmcode soll kommentiert und dokumentiert werden \\
\textbf{Beschreibung:} Der Programmcode soll kommentiert und dokumentiert werden, um eine effizientere Arbeitsweise im Verlauf des Projektes möglich zu machen. \\
\textbf{Quelle:} Treffen am 26.04.2018 \\
\textbf{Abhängigkeiten:} - \\\\

\textbf{R24.)} SpeicherungMovebankNutzerdaten \\
\textbf{Funktion:} Nutzerdaten für die Movebank sollen in den Einstellungen eingetragen werden können \\
\textbf{Beschreibung:} Die App soll in den Einstellungen eine Möglichkeit anbieten, Movebank Nutzerdaten eintragen zu können, um auch Daten aus der Movebank zu laden, die nur mit einem dortigen Account abrufbar sind. \\
\textbf{Quelle:} Treffen am 26.04.2018 \\
\textbf{Abhängigkeiten:} R4 \\\\
	


\end{document}