\documentclass[a4paper]{article}

\usepackage[utf8]{inputenc}

\title{Requirements}

\begin{document}
	
	\maketitle
	
	\begin{enumerate}
		\item Die App sollte auf Smartphones mit Android 5.1 oder höher, Internet-Zugang und einer armeabi, armeabi-v7a oder amr64-v8a Architektur lauffähig sein.
		\item Die App sollte bei fehlender Internetverbindung mit eingeschränkter Funktionalität weiter nutzbar sein
		\item Die Vorhersagen sollen visuell dargestellt werden
		\item Daten sollen aus der Movebank geladen werden
		\item Die vorhandenen Daten für einen gegebenen Vogel und Zeitraum sollen visualisiert werden
		\item Für einen gegebenen Vogel soll ein wahrscheinlicher Aufenthaltsraum berechnet werden
		\item Die Positionsvorhersage soll in Zonen mit unterschiedlicher Wahrscheinlichkeit unterteilt werden
		\item Die Positionsvorhersage soll für einen gegebenen Zeitraum berechnet werden
		\item Die Software soll modular gestaltet sein
		\item Die App soll über eine einheitliche Schnittstelle für verschiedene Vorhersagealgorithmen besitzen
		\item Es soll nach einem bestimmten Vogel gesucht werden können
		\item Bei WLAN-Zugang sollen die Daten in Cesium visualisiert werden.
		\item Bei Zugang zum mobielen Internet sollen die Daten 2-dimensional visualisiert werden
		\item Offline sollen die Daten falls offline-Karten verfügbar sind auf diesen visualisiert werden
		\item Zeitraum für die Vorhersage soll vom Nutzer eingestellt werden können
		\item Der Nutzer soll einstellen können welche der verfügbaren Daten in die Vorhersage mit einfließen
		\item Die App soll robust gegenüber formal falschen Daten aus der Movebank sein
		\item Der nutzer soll auf einen hohen Anteil formal falscher Daten hingewiesen werden
		\item Die Berechnungszeit der Vorhersage soll den vom Nutzer erwarteten Zeitraum nicht deutlich überschreiten
		\item Die App soll über eine Touch-basierte Benutzeroberfläche verfügen
		\item Die App soll für die aktuelle Vorhersage offline-Daten speichern
		\item Über mobile Daten soll die aktuelle Vorhersage aktualisiert werden können
		\item Code sollte kommentiert und dokumentiert werden
		
 	\end{enumerate}
	
\end{document}