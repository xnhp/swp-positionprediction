%%%%%%%%%%%%%%%%%%%%%%%%%%%%%%%%%%%%%%%%%
% Simple Sectioned Essay Template
% LaTeX Template
%
% This template has been downloaded from:
% http://www.latextemplates.com
%
% Note:
% The \lipsum[#] commands throughout this template generate dummy text
% to fill the template out. These commands should all be removed when 
% writing essay content.
%
%%%%%%%%%%%%%%%%%%%%%%%%%%%%%%%%%%%%%%%%%

%----------------------------------------------------------------------------------------
%	PACKAGES AND OTHER DOCUMENT CONFIGURATIONS
%----------------------------------------------------------------------------------------

\documentclass[12pt]{article} % Default font size is 12pt, it can be changed here

\usepackage{geometry} % Required to change the page size to A4
\geometry{a4paper} % Set the page size to be A4 as opposed to the default US Letter

\usepackage{tabularx}
\usepackage[utf8]{inputenc}


\usepackage{graphicx} % Required for including pictures

\usepackage{float} % Allows putting an [H] in \begin{figure} to specify the exact location of the figure
\usepackage{wrapfig} % Allows in-line images such as the example fish picture

\usepackage{lipsum} % Used for inserting dummy 'Lorem ipsum' text into the template

\usepackage{hyperref}

\usepackage[shortlabels]{enumitem}
                    
\usepackage{xcolor}

\linespread{1.2} % Line spacing

%\setlength\parindent{0pt} % Uncomment to remove all indentation from paragraphs

\graphicspath{{Pictures/}} % Specifies the directory where pictures are stored

\begin{document}
\setlist[enumerate, 1]{1\textsuperscript{o}}

%----------------------------------------------------------------------------------------
%	TITLE PAGE
%----------------------------------------------------------------------------------------

\begin{titlepage}

\newcommand{\HRule}{\rule{\linewidth}{0.5mm}} % Defines a new command for the horizontal lines, change thickness here

\center % Center everything on the page

\textsc{\Large Position Prediction Based on Movement Data}\\[0.5cm] % Major heading such as course name
\textsc{\large System Design Document}\\[0.5cm] % Minor heading such as course title

\vfill

\emph{Authors:}\\
Timo Jockers, Oliver Mänder, Benjamin Moser, \\
Manuel Prinz, Sebastian Strumbelj, Simon Suckut

\vfill % Fill the rest of the page with whitespace

\end{titlepage}

%----------------------------------------------------------------------------------------
%	TABLE OF CONTENTS
%----------------------------------------------------------------------------------------

\tableofcontents % Include a table of contents

\newpage % Begins the essay on a new page instead of on the same page as the table of contents 

%----------------------------------------------------------------------------------------
%	INTRODUCTION
%----------------------------------------------------------------------------------------

% Changelog
\section{Änderungen}


\begin{tabular}{|c|c|p{10cm}|c|}
\hline
Version & Author & Description of changes & Date \\ \hline\hline

%changelog content 
1.0 & SS & \color{red}{Aufbau der App in Android Studio} & x.5.2018 \\\hline
	&	&	&	\\\hline

\end{tabular}


{\small

\noindent
\\\\Legend: \\
SSJ = Sebastian Strumbelj \\
BM = Benjamin Moser \\
MP = Manuel Prinz \\
OM = Oliver Mänder \\
SS = Simon Suckut \\
TJ = Timo Jockers \\
ALL = SSJ, BM, MP, OM, SS, TJ \\
}


%------------------------------------------------

\section{Einleitung} % Major section

%------------------------------------------------

\subsection{Zweck}
Das Erstellen einer App, um einem Ornithologen das Finden eines Vogels zu erleichtern, indem sie dessen zukünftige Position mit einem Vorhersagemodell abschätzt. 

%------------------------------------------------

\subsection{Design-Ziele}

Diese Dokumentation spezifiert alle Eigenschaften, die unsere App benötigt, um deren Zweck zu Erfüllen. Um eine Vorhersage für Vögel zu treffen und dem Ornithologen bei seiner Suche zu unterstützen, sollte die App folgende Funktionen beinhalteten: 
\begin{itemize}
	\item Aufbereiten und Speichern der vorhandenden und benötigten Daten
	\item Vorhersageberechnung 
	\item Visualisierung der Daten
	\item Userinterface zur Interaktion 
\end{itemize}
Die App beinhaltet keine Funktionen die folgenden entsprechen:
\begin{itemize}
	\item Daten aus der Vergangenheit aufwendig visualisieren, um das vergangene Verhalten zu studieren
	\item \color{red}{TODO}

\end{itemize}



% This document fully specifies all traits relevant for the implementation of...

% The system does...

% The system does not...

\subsection{Definitionen und Abkürzungen}

	\paragraph{App, Applikation} Soweit nicht anders angemerkt wird hiermit die zu entwickelnde Android-Applikation bezeichnet.
	 \paragraph{Forscher} Personen, die wissenschaftlich das Verhalten von Vögeln untersuchen.
	 \paragraph{Mobile Daten} Ein Smartphone kann sich entweder über ein WiFi-Netzwerk oder über ein mobiles Datennetzwerk (3G, LTE, ...) mit dem Internet verbinden. 
	\paragraph{Vorhersage-Modell/Algorithmus} Die zur Berechnung einer Vorhersage verwendete Methodik. 
	\paragraph{API} \textit{Application Programming Interface}, hier insbesondere Schnittstellen mit Providern wie \textit{Movebank} oder \textit{Cesium}. 
	\paragraph{Outdoor-Modus, Normal-Modus} Die Applikation verfügt über zwei verschiedene Funktionsmodi. Einer davon ist der sog. Outdoor-Modus.
	\paragraph{Cesium} Ist eine Javascript-Bibliothek für die Arbeit mit dreidimensionalen Kartendaten.
	\paragraph{Offline-Karten} Ist ein 2D-Karten-Provider der Kartendaten liefert, wenn keine Internetverbindung besteht.

	\paragraph{SRS} Software Requirements Document



\subsection{Referenzen}
\begin{itemize}
	\item SRS
	\item Besprechungen mit den Betreuern
	\item Besprechungen mit einem Forscher
	\item \href{https://git.uni-konstanz.de/kn/swp2018/group12/tree/master/Dokumentation/Lizenzen}{Notizen zu Recherche über Nutzungsbedingungen und Lizenzen von Systemkomponenten}.
\end{itemize}


\subsection{Überblick}

Der erste Teil des Dokumentes beschäftigt sich mit vorhandenden 


% If there is a system to be replaced.
% Otherwise: comparable systems, competing systems.


% The first part gives an overview of this document. The second part will give a description of the current architecture. In the third part will be a description of the architecture proposed by MKS-Software Solutions.

% -------------------------------------------------------------------


\section{Aktuelle Architektur}

% Wird wahrscheinlich wegfallen

% If there is a system to be replaced.
% Otherwise: comparable systems, competing systems

% -------------------------------------------------------------------

% Formulierung?
\section{Vorgeschlagene Architektur}

\subsection{Überblick}

% Textbeschreibung, Component Diagram

\subsection{Teilsysteme}

% The sub-subsystems are as denoted in Figure \ref{fig:component-diagram}.
% It follows a detailed description of the subsystems and their tasks.

% \paragraph{Individual Databases, DBDriver} will be part of a
% Database Management System such as MySQL.

% etc

% -------------------------------------------------------------------


\subsection{Hardware-Software-Zuordnung}

% \begin{figure}
% 	\includegraphics[width=\textwidth]{pictures/deployment-diagram.png}
% 	\caption{Deployment Diagram}
% 	\label{fig:deployment-diagram}
% \end{figure}

% Textbeschreibung


% -------------------------------------------------------------------

% Formulierung
\subsection{Management von persistenten Daten}

Um Daten nach dem Schließen der Applikation zu behalten, die Menge des Datenvolumens für Anfragen an die Movebank zu reduzieren und um Daten leichter verwalten zu können verfügt die Applikation über eine Schnittstelle zum Parsen und Speichern von XML-Dateien und eine SQLite-Datenbank. Außerdem werden Karten für den Offline-Modus lokal gespeichert.

\begin{itemize}
	\item \textit{SQLite-Datenbank}: In dieser Datenbank werden Daten von der Movebank lokal zwischen gespeichert um die Menge an Anfragen an die Movebank so gering wie möglich zu halten, die Menge des verbrauchten Datenvolumens zu reduzieren und die Daten leichter verwalten zu können.
	
	\item \textit{XML-Parser}: In der Applikation getätigte Einstellungen werden persistent in einer XML-Datei gespeichert.
	
	\item \textit{Offline-Karten}: Karten-Daten werden in dem vom Kartenanbieter verwendeten Format lokal zwischengespeichert.
	
	
\end{itemize}

% -------------------------------------------------------------------


\subsection{Sicherheit}

% Übersetzung?
\subsection{Global Software Control}

% • How will control signals be used?
% • How do the subsystems communicate? --> interface descriptions?
% • How do the subsystems synchronize? --> statechart diagram? Concurrency


\subsubsection{Interfaces}
\label{sec:interfaces}


% ----

% Übersetzung?
\subsection{Boundary Conditions}

% Sequenzdiagramm?
% \includegraphics[width = 0.8\textwidth]{pictures/SeqDia.png}

% Startup behaviour (What happens, when we start the system?)
% • Shutdown behaviour (What happens, when we stop the system?)
% • Failure behaviour (What happens, when the system
% missbehaves? Which kind of failures can you imagine? What
% does the system need to do then?)



\section{Anhänge}

\subsection{Geänderte Anforderungen}

% List changed requirements in old and new form.

%----------------------------------------------------------------------------------------

\end{document}