%%%%%%%%%%%%%%%%%%%%%%%%%%%%%%%%%%%%%%%%%
% Simple Sectioned Essay Template
% LaTeX Template
%
% This template has been downloaded from:
% http://www.latextemplates.com
%
% Note:
% The \lipsum[#] commands throughout this template generate dummy text
% to fill the template out. These commands should all be removed when 
% writing essay content.
%
%%%%%%%%%%%%%%%%%%%%%%%%%%%%%%%%%%%%%%%%%

%----------------------------------------------------------------------------------------
%	PACKAGES AND OTHER DOCUMENT CONFIGURATIONS
%----------------------------------------------------------------------------------------

\documentclass[12pt]{article} % Default font size is 12pt, it can be changed here

\usepackage{geometry} % Required to change the page size to A4
\geometry{a4paper} % Set the page size to be A4 as opposed to the default US Letter

\usepackage{tabularx}



\usepackage{graphicx} % Required for including pictures

\usepackage{float} % Allows putting an [H] in \begin{figure} to specify the exact location of the figure
\usepackage{wrapfig} % Allows in-line images such as the example fish picture

\usepackage{lipsum} % Used for inserting dummy 'Lorem ipsum' text into the template

\usepackage[shortlabels]{enumitem}
                    
\usepackage{xcolor}

\linespread{1.2} % Line spacing

%\setlength\parindent{0pt} % Uncomment to remove all indentation from paragraphs

\graphicspath{{Pictures/}} % Specifies the directory where pictures are stored

\begin{document}
\setlist[enumerate, 1]{1\textsuperscript{o}}

%----------------------------------------------------------------------------------------
%	TITLE PAGE
%----------------------------------------------------------------------------------------

\begin{titlepage}

\newcommand{\HRule}{\rule{\linewidth}{0.5mm}} % Defines a new command for the horizontal lines, change thickness here

\center % Center everything on the page

\textsc{\Large Position Prediction Based on Movement Data}\\[0.5cm] % Major heading such as course name
\textsc{\large System Design Document}\\[0.5cm] % Minor heading such as course title

\vfill

\emph{Authors:}\\
Timo Jockers, Oliver Mänder, Benjamin Moser, \\
Manuel Prinz, Sebastian Strumbelj, Simon Suckut

\vfill % Fill the rest of the page with whitespace

\end{titlepage}

%----------------------------------------------------------------------------------------
%	TABLE OF CONTENTS
%----------------------------------------------------------------------------------------

\tableofcontents % Include a table of contents

\newpage % Begins the essay on a new page instead of on the same page as the table of contents 

%----------------------------------------------------------------------------------------
%	INTRODUCTION
%----------------------------------------------------------------------------------------

% Changelog
\section{Änderungen}

% \begin{tabular}{|c|c|p{10cm}|c|}
% \hline
% Version & Author & Description of changes & Date \\
% \hline\hline
% 1.0 & M,S,R & Inital release & 1.1.2018 \\\hline
% 1.1 & M & Merged modules Order- and Animal-Reporting(*) & 5.1.2018 \\\hline
% 1.2 & S & New module 'Controlling'(**) & 6.1.2018 \\\hline
% 1.3 & M & Modified requirements (R8), (R2) after phone call with Zoo Director & 12.1.2018 \\\hline
% \end{tabular}
% \\\\
% {\small
% Legend: \\
% S = Sebastian Strumbelj \\
% M = Benjamin Moser \\
% R = Simon Reufsteck \\
% }

%------------------------------------------------

\section{Einleitung} % Major section

%------------------------------------------------

\subsection{Zweck}



%------------------------------------------------

\subsection{Design-Ziele}

% This document fully specifies all traits relevant for the implementation of...

% The system does...

% The system does not...

\subsection{Definitionen und Abkürzungen}

% Hier relevante Punkte aus SRS übernehmen?

\paragraph{SRS} Software Requirements Document



\subsection{Referenzen}


\subsection{Überblick}

% If there is a system to be replaced.
% Otherwise: comparable systems, competing systems.


% -------------------------------------------------------------------


\section{Aktuelle Architektur}

% Wird wahrscheinlich wegfallen

% If there is a system to be replaced.
% Otherwise: comparable systems, competing systems

% -------------------------------------------------------------------

% Formulierung?
\section{Vorgeschlagene Architektur}

\subsection{Überblick}

% Textbeschreibung, Component Diagram

\subsection{Teilsysteme}

% The sub-subsystems are as denoted in Figure \ref{fig:component-diagram}.
% It follows a detailed description of the subsystems and their tasks.

% \paragraph{Individual Databases, DBDriver} will be part of a
% Database Management System such as MySQL.

% etc

% -------------------------------------------------------------------


\subsection{Hardware-Software-Zuordnung}

% \begin{figure}
% 	\includegraphics[width=\textwidth]{pictures/deployment-diagram.png}
% 	\caption{Deployment Diagram}
% 	\label{fig:deployment-diagram}
% \end{figure}

% Textbeschreibung


% -------------------------------------------------------------------

% Formulierung
\subsection{Management von persistenten Daten}


% -------------------------------------------------------------------


\subsection{Sicherheit}

% Übersetzung?
\subsection{Global Software Control}

% • How will control signals be used?
% • How do the subsystems communicate? --> interface descriptions?
% • How do the subsystems synchronize? --> statechart diagram? Concurrency


\subsubsection{Interfaces}
\label{sec:interfaces}


% ----

% Übersetzung?
\subsection{Boundary Conditions}

% Sequenzdiagramm?
% \includegraphics[width = 0.8\textwidth]{pictures/SeqDia.png}

% Startup behaviour (What happens, when we start the system?)
% • Shutdown behaviour (What happens, when we stop the system?)
% • Failure behaviour (What happens, when the system
% missbehaves? Which kind of failures can you imagine? What
% does the system need to do then?)



\section{Anhänge}

\subsection{Geänderte Anforderungen}

% List changed requirements in old and new form.

%----------------------------------------------------------------------------------------

\end{document}