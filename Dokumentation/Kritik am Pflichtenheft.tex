\documentclass{article}

\usepackage[utf8]{inputenc}

\begin{document}
	
	\subsection*{Kretik am Pflichtenheft:}
	
	\paragraph{Definitionen:}
	
	Movebank und Cesium sollten noch mit aufgenommen werdne
	
	\paragraph{Nutzercharakterisierung:}
	
	Nutzer nicht auf Forscher am Max-Plank Institut beschränken. Nicht so tiefen technischen Kenntnissen des Nutzers ausgehen
	
	\paragraph{Szenario:}
	
	Nicht von tatsächlichem Antreffen des Vogels ausgehen\\
	Muss das Feedback des Forsches mit in das Szenario mit rein? (eventuell für spätere Erweiterungen)
	Tracking der Person die nach de Vogel sucht? (als weiterführende Idee)
	
	\paragraph{Annahmen und Abhängigkeiten:}
	
	Schnittstellen zu jeder Zeit online?
	
	\paragraph{Requirements:} Formulierung bei "gegeben ein Vogel"\\
	"Jeder Punkt in der Umgebung" abschwächen\\
	(R12) nur Kurzer Zeitraum vor der Vorhersage/Zeitraum beschränken\\
	(R8) Nutzer soll \textbf{Berechnung} der Vorhersage justieren können\\
	Google maps/leavelet und OpeStreetMap/OSMDroid statt Mapbox\\
	(R14) Keinen speziellen Karten-Provider an geben (höchstens als Beispiel)\\
	"Falsch formatierte Daten" statt "formal falsche Daten"\\
	Performance der Hardware und Komplexität der Algorithmen sollte bei den Wartezeiten erwähnt werden\\
	(R19) was kommt als Antwort auf die Suche?\\
	(R20) Falsche Daten sollen ausgelassen werden\\
	(R21) arm64-v8a\\
	
	
	\paragraph{Optionale Requirements:}
	(O1) "Virtual Reality-Umgesbung" statt "Virtual Reality-Raum"\\
	(O2) Wetter, Temperatur und weitere mögliche Parameter erwähnen\\
	
	
	\paragraph{Milestone 1}
	Cesium zu komplex für ersten Milestone\\
	
	\paragraph{Weitere Kritik:}
	
	Zeitplan detailierter aufteilen\\
	Testen sollte in den Zeitplan mit aufgenommen werden\\
	
	
	
	
\end{document}