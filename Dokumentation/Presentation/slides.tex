\documentclass{beamer}
\usepackage[utf8]{inputenc} 

\usepackage{textcomp}

\usepackage{caption}

\usepackage{anyfontsize}

\usepackage[font=itshape]{quoting}
\quotingsetup{vskip=4pt} % vertical spacing of quotes


\usecolortheme{seahorse} %colour theme, cf https://www.hartwork.org/beamer-theme-matrix/

\title[] {Group 1.2 - Predicition App}
\date{} % no date

% empty to have to numbering of split-up frames from allowframebreaks
% cf http://tex.stackexchange.com/questions/193308/how-can-we-change-allowframebreaks-numbering-in-the-title#212292
\setbeamertemplate{frametitle continuation}{}


\begin{document}

\frame{\titlepage}

\begin{frame}
	\frametitle{What we have achieved so far - Design}
    


\end{frame}




\begin{frame}
	\frametitle{What we have achieved so far - Movebank}
    

    
\end{frame}





\begin{frame}
	\frametitle{What we have achieved so far - Database}
    

    
\end{frame}





\begin{frame}
	\frametitle{What we have achieved so far - Algorithms I}
	\Large{AlgorithmExtrapolationExtended}
	\normalsize{}
			\begin{itemize}
				\item[(+)] Good if the variance of the angles is not too big
				\item[(+)] Later datapoints are weighted more
				\item[(+)] Fast
				\item[(+)] Easy to understand
				\item[] 
				\item[( - )] Not very accurate
				\item[( - )] Early data gets ignored

			\end{itemize}
\end{frame}

\begin{frame}
	\frametitle{What we have achieved so far - Algorithms II}
	\Large{AlgorithmSimilarTrajectory}
	\normalsize{}
			\begin{itemize}
				\item[(+)] Good if the measuring frequency is high
				\item[(+)] Later datapoints are important for the result
				\item[(+)] Multiple trajectories can be found
				\item[(+)] Easy to understand
				\item[] 
				\item[( - )] Frequency is not always high $\Rightarrow$ Wrong result
				\item[( - )] Higher complexity than the other algorithm
			\end{itemize}
\end{frame}














\begin{frame}
	\frametitle{What we have achieved so far - Visualization}
    

    
\end{frame}









\end{document}