\documentclass{article}

\usepackage[utf8]{inputenc}

\usepackage{soulutf8}

\setlength{\parindent}{0pt} %verhindert das Einrücken zu Beginn eines Absatzes


\begin{document}
	



	% siehe auch Seite im Repository-Wiki
	% https://git.uni-konstanz.de/kn/swp2018/group12/wikis/milestones





	\section*{Definition der Milestones, Projektzeitplan}
	
	\subsection*{Milestone 1}
	Fertigstellung am 15.5.
	\begin{itemize} 
		\item \hl{Software-Architektur festgelegt?}
		\item \hl{Software-Design festgelegt?}	
	 	 \item Ein Grundgerüst der Applikation ist fertiggestellt: Es sind folgende Views funktional umgesetzt:
	 	 \begin{itemize} 
	 	  	 \item Caesium (eine Kartenumgebung wird angezeigt, Navigation möglich)
	 	  	 \item Mapbox (eine Kartenumgebung wird angezeigt, Navigation möglich)
	 	  	 \item Diverse UI-Elemente (mindestens Eingabefeld, Button)
	 	 \end{itemize} 
	\end{itemize} 

	\vspace{1em}

	\subsection*{Milestone 2}
	Fertigstellung am 5.6.
	\begin{itemize} 
	 	 \item Es wird eine Vorhersage mittels eines einfachen Vorhersagemodells berechnet.
	 	 \item Es existiert eine funktionale Visualisierung des Vorhersage-Ergebnisses.
	 	 \item Die Positionsdaten für die Vorhersage werden von der \textit{Movebank}-API bezogen.
	 	 \item Der Aufbau des User-Interfaces ist festgelegt (nicht-funktionale Mockups).
	\end{itemize}

	\vspace{1em}

	\subsection*{Milestone 3}
	Fertigstellung am 10.7.
	\begin{itemize} 
	 	 \item Anforderungen, die mit eingeschränktem Internetzugriff in Verbindung stehen sind realisiert.
	 	 \item Die momentan angezeigte Vorhersage kann aktualisiert werden (Prüfung, ob neue Daten verfügbar, Neuberechnung; auch unter Verwendung von lediglich mobilem Datennetz)
	 	 \item Die Visualisierung ist vollständig und auf gute Interpretierbarkeit und Ästhetik optimiert.
	 	 \item Das User Interface ist vollständig und auf gute Benutzbarkeit und Ästhetik optimiert.
	\end{itemize} 



\end{document}