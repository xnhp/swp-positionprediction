
%----------------------------------------------------------------------------------------
%	PACKAGES AND OTHER DOCUMENT CONFIGURATIONS
%----------------------------------------------------------------------------------------

\documentclass[12pt]{article} % Default font size is 12pt, it can be changed here

\usepackage[utf8]{inputenc}

\usepackage{geometry} % Required to change the page size to A4
\geometry{a4paper} % Set the page size to be A4 as opposed to the default US Letter

\usepackage{graphicx} % Required for including pictures

\usepackage{float} % Allows putting an [H] in \begin{figure} to specify the exact location of the figure
\usepackage{wrapfig} % Allows in-line images such as the example fish picture

\usepackage{lipsum} % Used for inserting dummy 'Lorem ipsum' text into the template

\usepackage[shortlabels]{enumitem}
                    
\usepackage{xcolor}

% text-marker-like highlighting of text
% https://tex.stackexchange.com/questions/141569/highlight-textcolor-and-boldface-simultaneously#141572
\usepackage{soulutf8}

\usepackage{hyperref}

\usepackage{glossaries}
\makeglossaries

\linespread{1.2} % Line spacing

\setlength\parindent{0em} % Uncomment to remove all indentation from paragraphs

\setlength{\parskip}{1.25ex}

\graphicspath{{Pictures/}} % Specifies the directory where pictures are stored

\begin{document}

\setlist[enumerate, 1]{1\textsuperscript{o}}

%----------------------------------------------------------------------------------------
%	TITLE PAGE
%----------------------------------------------------------------------------------------

\begin{titlepage}

\newcommand{\HRule}{\rule{\linewidth}{0.5mm}} % Defines a new command for the horizontal lines, change thickness here

\center % Center everything on the page

\textsc{\Large Position Prediction based on Movement Data}\\[0.5cm] % Major heading such as course name
\textsc{\large Software Requirements Specification}\\[0.5cm] % Minor heading such as course title

\vfill

\emph{Authors:}\\
...

\vfill % Fill the rest of the page with whitespace

\end{titlepage}

%----------------------------------------------------------------------------------------
%	TABLE OF CONTENTS
%----------------------------------------------------------------------------------------

\tableofcontents % Include a table of contents

\newpage % Begins the essay on a new page instead of on the same page as the table of contents 

%----------------------------------------------------------------------------------------
%	INTRODUCTION
%----------------------------------------------------------------------------------------

\section{Einleitung}

%------------------------------------------------

\subsection{Zweck}


% Why do we write an SRS?
% Whom do we write this document for?

 Dieses Dokument dient  zur Fixierung der erhobenen Anforderungen und Einschränkungen an die zu entwickelnde Applikation. Damit liefert es eine Grundlage und Referenz für den Entwicklungsprozess sowie für die Überprüfung des Systems auf Vollständigkeit nach Fertigstellung. Für den Kunden legt dieses Dokument fest, welcher Funktionsumfang realisiert werden wird.

%------------------------------------------------

% "Scope"
\subsection{Funktionsumfang}

% State the product to be produced
% Take care to make this consistent with other documents.
Die zu entwickelnde Android-Applikation trifft, basierend auf Informationen aus der Vergangenheit, eine Vorhersage über den Aufenthaltsort eines Vogels zu einem gegebenen Zeitpunkt in der Zukunft. 

Dabei kann der Nutzer wählen, welche der verfügbaren Daten und Verfahren in die Vorhersage einfließen. 

Das Ergebnis der Vorhersage wird in einer zwei- oder dreidimensionalen Kartenlandschaft visualisiert.

\hl{Weiterhin ist integriert eine VR-Visualisierung. (Zu klären!)}


\subsection{Definitionen und Abkürzungen}

% Könnte man machen, evtl. zu Umständlich
% \newglossaryentry{forscher}{name=Forscher, description={%
% 	Ein Mitarbeiter des Max-Planck-Instituts und Nutzer der App.
% }}

% \newglossaryentry{utc}{name=UTC, description={Coordinated Universal Time}}
% \newglossaryentry{adt}{name=ADT, description={Atlantic Daylight Time}}
% \newglossaryentry{est}{name=EST, description={Eastern Standard Time}}

% \printglossaries
 
% % Use the terms
% \gls{utc} is 3 hours behind \gls{adt} and 10 hours ahead of \gls{est}.

\paragraph{App, Applikation} Soweit nicht anders angemerkt wird hiermit die zu entwickelnde Android-Applikation bezeichnet.
 \paragraph{Forscher} Ein Mitarbeiter des Max-Planck-Instituts und Nutzer der App.
 \paragraph{Mobile Daten} Ein Smartphone kann sich entweder über ein WiFi-Netzwerk oder über ein mobiles Datennetzwerk (3G, LTE, ...) mit dem Internet verbinden. 
\paragraph{Vorhersage-Modell/Algorithmus} Die zur Berechnung einer Vorhersage verwendete Methodik. 

\subsection{Referenzen}

\begin{itemize} 
 	 \item  \href{https://docs.google.com/document/d/1Yc2f18JFaHyhrgM2h2WiATQ0zVmZnsc9W1ImhwWJF-g/edit?usp=sharing}{Begleitdokument zum Softwareprojekt SS 2018}
 	 \item Projektzeitplan und Definition der Milestones
\end{itemize} 



\subsection{Überblick}

% Explain, how your document will proceed. (Tell them, what
% you will tell them).

% This document is made up of three chapters. This first chapter describes
% the scope and purpose of this document, used definitions, acronyms and
% abbrevians. It follows a list of references.
% \noindent
% In the second chapter, a rough overview of the system to be developed is
% given.
% \noindent
% The third chapter lists the requirements in a rigorous manner according
% to the IEEE Standard. Use-case and sequence diagrams are used to
% illustrate relationships.

\section{Beschreibung des Produkts}

\subsection{Perspektive}
% Relation to other products (e.g. overall system ⇒ Class diagram?)
% • Known interfaces (e.g. Mailserver)
...



% [text zum Funktionsumfang]

% Movebank, Movebank Livefeeds, Cesium

% Rough description of what the system will be able to do.
% The system shall become a centralised means to manage business processes
% that are central to the zoo's daily business.
% \noindent
% It shall work on top of and interface with the already present hard- and
% software infrastructure.
% \noindent
% The is no such system currently in use.

...

% User Characteristics
\subsection{Charakterisierung der Nutzer}

% Die Nutzer der Applikation sind Forscher am Max-Planck-Institut.
% Diese besitzen ein gutes Verständnis für abstrakte und technische Konzepte und verfügen über gutes bis sehr gutes technisches Fachwissen.

% The users of the software will adhere to one of the following
% archetypes:

% Describe the user: knowledge, (computer) education of users
% • The kind of users you expect (trained experts, users without
% knowledge)

% \paragraph{Zookeepers} The employees who work on tasks that are not
% administrative. These persons can be assumed to have average level
% computer skills. However, since their main place of work is around the
% zoo and with the animals, they might not enjoy indulging in long and
% complicated interactions with a stationary digital system.

% \paragraph{Executive Managers} The people who handle major ecological
% and ideological decisions. These people are assumed to have average to
% above-average computer skills, as well as some preference to abstract
% thinking.
% % zoo director

% \paragraph{Business Administration} People who manage day-to-day
% business processes. These people are assumed to be used to regular work
% with computers, however their proficiency might still be basic. 
% % secretary 

\subsection{Constraints}
% Things that could constraint the development
% • Law, Company internals, hardware limitations, safety
% criticality, ...

% \begin{itemize} 
%  	 \item \textbf{Lizenzen} 
%  	 \item API-Limits? (Movebank, Cesium)
%  	 \item (Eingeschränkt) Nutzbar ohne Internetverbindung
%  	 \item schlechte Sicht
% \end{itemize} 

% The system shall adhere to known standards and built systems and be
% strictly on budget (TODO: Reference to "no fancy new expensive stuff").
% \noindent
% The system shall always enforce the wealth of the animal beings.
% \subsection{Assumptions and Dependencies}

% Any other things, that have to be mentioned (e.g. external
% schemata or similar).

% tablets, tablet apps are given/do not have to be developed
% messaging system is given
% list of feed vendors is given

% The following systems are assumed to be present and able to interface
% with the new management software.
% \begin{itemize} 
% 	\item There is an identification and authentification system in place.
%  	\item  There is a messaging system already in place that can be used
%  	 to send text messages and data reports to one or more employees.
%  	 \item Basic hard- and software infrastructure is present. The newly
%  	 developed system will run as applications on that infrastructure.
% \end{itemize}


\section{Specific Requirements}

% struktur korrekt? \\

% class diagram, use case diagram, seq diag

\subsection{Functions}

\begin{enumerate}[(R1)]
		\item Die App sollte auf Smartphones mit Android 5.1 oder höher, Internet-Zugang und einer armeabi, armeabi-v7a oder amr64-v8a Architektur lauffähig sein. \\
		Function: Lorem ipsum dolor sit amet, consectetur adipisicing elit, sed do eiusmod \\
		Description:  Lorem ipsum dolor sit amet, consectetur adipisicing elit, sed do \\
		Source:  Lorem ipsum dolor sit amet, consectetur adipisicing elit, sed do djka qwj jskd. \\
		Dependency: -
		\item Die App sollte bei fehlender Internetverbindung mit eingeschränkter Funktionalität weiter nutzbar sein
		\item Die Vorhersagen sollen visuell dargestellt werden
		\item Daten sollen aus der Movebank geladen werden
		\item Die vorhandenen Daten für einen gegebenen Vogel und Zeitraum sollen visualisiert werden
		\item Für einen gegebenen Vogel soll ein wahrscheinlicher Aufenthaltsraum berechnet werden
		\item Die Positionsvorhersage soll in Zonen mit unterschiedlicher Wahrscheinlichkeit unterteilt werden
		\item Die Positionsvorhersage soll für einen gegebenen Zeitraum berechnet werden
		\item Die Software soll modular gestaltet sein
		\item Die App soll über eine einheitliche Schnittstelle für verschiedene Vorhersagealgorithmen besitzen
		\item Es soll nach einem bestimmten Vogel gesucht werden können
		\item Bei WLAN-Zugang sollen die Daten in Cesium visualisiert werden.
		\item Bei Zugang zum mobielen Internet sollen die Daten 2-dimensional visualisiert werden
		\item Offline sollen die Daten falls offline-Karten verfügbar sind auf diesen visualisiert werden
		\item Zeitraum für die Vorhersage soll vom Nutzer eingestellt werden können
		\item Der Nutzer soll einstellen können welche der verfügbaren Daten in die Vorhersage mit einfließen
		\item Die App soll robust gegenüber formal falschen Daten aus der Movebank sein
		\item Der nutzer soll auf einen hohen Anteil formal falscher Daten hingewiesen werden
		\item Die Berechnungszeit der Vorhersage soll den vom Nutzer erwarteten Zeitraum nicht deutlich überschreiten
		\item Die App soll über eine Touch-basierte Benutzeroberfläche verfügen
		\item Die App soll für die aktuelle Vorhersage offline-Daten speichern
		\item Über mobile Daten soll die aktuelle Vorhersage aktualisiert werden können
		\item Code sollte kommentiert und dokumentiert werden
		\item Nutzerdaten für die Movebank sollen in den Einstellungen eingetragen werden können	
 	\end{enumerate}
	

% \begin{enumerate}[(R1)] 
%  	\item Ordering new feed \\
% 	Function: The system shall be able to order new feed from the chosen dealer if necessary. \\
% 	Description: If there is no feed, the zookeepers shall be able to order new feed. \\
% 	Source: Description of the zoo keeper \\
% 	Dependency: -
%  \end{enumerate}




\section{Anhänge}

...

\end{document}