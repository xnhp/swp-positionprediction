
%----------------------------------------------------------------------------------------
%	PACKAGES AND OTHER DOCUMENT CONFIGURATIONS
%----------------------------------------------------------------------------------------

\documentclass[12pt]{article} % Default font size is 12pt, it can be changed here

\usepackage[utf8]{inputenc}

\usepackage{geometry} % Required to change the page size to A4
\geometry{a4paper} % Set the page size to be A4 as opposed to the default US Letter

\usepackage{graphicx} % Required for including pictures

\usepackage{float} % Allows putting an [H] in \begin{figure} to specify the exact location of the figure
\usepackage{wrapfig} % Allows in-line images such as the example fish picture

\usepackage{lipsum} % Used for inserting dummy 'Lorem ipsum' text into the template

\usepackage[shortlabels]{enumitem}
                    
\usepackage{xcolor}

% text-marker-like highlighting of text
% https://tex.stackexchange.com/questions/141569/highlight-textcolor-and-boldface-simultaneously#141572
\usepackage{soulutf8}

\usepackage{hyperref}

\usepackage{glossaries}
\makeglossaries

\linespread{1.2} % Line spacing

\setlength\parindent{0em} % Uncomment to remove all indentation from paragraphs

\setlength{\parskip}{1.25ex}

\graphicspath{{Pictures/}} % Specifies the directory where pictures are stored


% enumerate list with less spacing
% usage:
% \begin{cptenumerate}
% 	\advantageit Is a very good one
% 	\disadvantageit Is a very, very complex one.
% \end{cptenumerate}
\newenvironment{cptenumerate}[1][label=\arabic*.]{\begin{enumerate}[#1] \setlength\itemsep{0em}}{\end{enumerate}}

\newenvironment{cptitemize}[1][,]{\begin{itemize} \setlength\itemsep{0em}}{\end{itemize}}


\begin{document}

\setlist[enumerate, 1]{1\textsuperscript{o}}

\begin{titlepage}

\newcommand{\HRule}{\rule{\linewidth}{0.5mm}} % Defines a new command for the horizontal lines, change thickness here

\center % Center everything on the page

\textsc{\Large Position Prediction based on Movement Data}\\[0.5cm] % Major heading such as course name
\textsc{\large Software Requirements Specification}\\[0.5cm] % Minor heading such as course title

\vfill

\emph{Autoren}\\
Timo Jockers, Oliver Mänder, Benjamin Moser,\\Manuel Prinz, Sebastian Strumbelj, Simon Suckut

\vfill % Fill the rest of the page with whitespace

\end{titlepage}

%================================================================================================================

\tableofcontents % Include a table of contents

\newpage

%================================================================================================================

\section{Einleitung} \label{einleitung}


\subsection{Überblick}

% Explain, how your document will proceed. (Tell them, what
% you will tell them).
Dieses Dokument ist in fünf Abschnitte aufgeteilt. Im ersten Abschnitt (Kapitel \ref{einleitung}) wird der Zweck dieses Dokuments erläutert und der Funktionsumfang der Applikation grob beschrieben. Desweiteren werden verwendete Definitionen und Abkürzungen sowie weitere referenzierte Dokumente aufgeführt. Im zweiten Abschnitt (Kapitel \ref{kontext}) wird der Kontext beschrieben, in dem das Produkt verwendet werden soll. Darunter fallen das Verhältnis zu anderen Produkten, die Nutzercharakteristika sowie mögliche Abhängigkeiten und Einschränkungen.
Der dritte Abschnitt (Kapitel \ref{mindestanforderungen} bis \ref{nicht-anforderungen}) enthält eine detaillierte Auflistung der Anforderungen. Im vierten Abschnitt (Kapitel \ref{zeitplan}) wird der Projektzeitplan, insbesondere die Meilensteine, festgelegt. Der letzte Abschnitt (Kapitel \ref{diagramme}) enthält die wesentlichen UML-Diagramme.
%------------------------------------------------

\subsection{Zweck dieses Dokumentes}


% Why do we write an SRS?
% Whom do we write this document for?

 Dieses Dokument dient  zur Fixierung der erhobenen Anforderungen und Einschränkungen an die zu entwickelnde Applikation. Damit liefert es eine Grundlage und Referenz für den Entwicklungsprozess sowie für die Überprüfung des Systems auf Vollständigkeit nach Fertigstellung. Für den Kunden legt dieses Dokument fest, welcher Funktionsumfang realisiert werden wird.

%------------------------------------------------

% "Scope"
\subsection{Produktbeschreibung}

% State the product to be produced
% Take care to make this consistent with other documents.
Die zu entwickelnde Android-Applikation trifft, basierend auf Informationen aus der Vergangenheit, eine Vorhersage über den Aufenthaltsort eines Vogels zu einem gegebenen Zeitpunkt in der Zukunft. 

Dabei kann der Nutzer wählen, welche der verfügbaren Daten und Verfahren in die Vorhersage einfließen. 

Das Ergebnis der Vorhersage wird in einer zwei- oder dreidimensionalen Kartenlandschaft visualisiert.

\hl{Weiterhin ist integriert eine VR-Visualisierung. (Zu klären!)}


\subsection{Definitionen und Abkürzungen}

% Könnte man machen, evtl. zu Umständlich
% \newglossaryentry{forscher}{name=Forscher, description={%
% 	Ein Mitarbeiter des Max-Planck-Instituts und Nutzer der App.
% }}

% \newglossaryentry{utc}{name=UTC, description={Coordinated Universal Time}}
% \newglossaryentry{adt}{name=ADT, description={Atlantic Daylight Time}}
% \newglossaryentry{est}{name=EST, description={Eastern Standard Time}}

% \printglossaries
 
% % Use the terms
% \gls{utc} is 3 hours behind \gls{adt} and 10 hours ahead of \gls{est}.

\paragraph{App, Applikation} Soweit nicht anders angemerkt wird hiermit die zu entwickelnde Android-Applikation bezeichnet.
 \paragraph{Forscher} Ein Mitarbeiter des Max-Planck-Instituts und Nutzer der App.
 \paragraph{Mobile Daten} Ein Smartphone kann sich entweder über ein WiFi-Netzwerk oder über ein mobiles Datennetzwerk (3G, LTE, ...) mit dem Internet verbinden. 
\paragraph{Vorhersage-Modell/Algorithmus} Die zur Berechnung einer Vorhersage verwendete Methodik. 
\paragraph{API} \textit{Application Programming Interface}, hier insbesondere Schnittstellen mit Providern wie \textit{Movebank} oder \textit{Caesium}. 

\subsection{Referenzen}

\begin{itemize} 
 	 \item  \href{https://docs.google.com/document/d/1Yc2f18JFaHyhrgM2h2WiATQ0zVmZnsc9W1ImhwWJF-g/edit?usp=sharing}{Begleitdokument zum Softwareprojekt SS 2018}
\end{itemize} 



%================================================================================================================

\newpage
\section{Produktkontext} \label{kontext}

\subsection{Externe Produkte}
% Relation to other products (e.g. overall system ⇒ Class diagram?)
% • Known interfaces (e.g. Mailserver)
Die Applikation soll Forschern eine Vorhersage bieten können, wo sie einen Vogel mit einer gewissen Wahrscheinlichkeit in naher Zukunft antreffen werden. 

In der Vergangenheit aufgezeichnete Positionsdaten werden dabei von \textit{Movebank} bezogen. Das Ergebnis der Vorhersage wird mittels \textit{Caesium} und \textit{Mapbox} visualisiert.


% User Characteristics
\subsection{Nutzercharakterisierung}

Die Nutzer der Applikation sind Forscher am Max-Planck-Institut für Ornithologie in Radolfzell.

Sie besitzen ein gutes Verständnis für abstrakte und technische Konzepte und verfügen über gutes bis sehr gutes Fachwissen in der Bedienung bzw. Programmierung digitaler Systeme.

Der Nutzer verfügt über mindestens gute Erfahrung im Umgang mit Smartphones und deren Bedienung.

Der Nutzer hat im Allgemeinen eine Vorstellung vom Umfang der verwendeten Datenmengen und der Komplexität der Analyse und ist sich bewusst dass eventuell bei Berechnungsvorgängen kürzere Wartezeiten auftreten können.
% zu vage?

Der Nutzer hat Erfahrung im Umgang mit geographischen Daten und deren Interpretation.

Der Nutzer hat grundliegende Erfahrung in der Interpretation von Visualisierungen.
% evtl. zu offensichtlich


\subsection{Annahmen und Abhängigkeiten}

% \begin{itemize} 
%  	 \item vorhersage-modelle
%  	 \item datenbank vorhanden
%  	 \item movebank erreichbar 
%  	 \item vogel-id/name vorhanden
% \end{itemize} 

\begin{itemize} 
 	 \item Die verwendeten Schnittstellen (\textit{Movebank}, \textit{Caesium}, ...) sind zu jedem Zeitpunkt verfügbar.
 	 \item Eine für die Abfrage der Daten aus \textit{Movebank} ausreichende Identifikation des gesuchten Vogels ist vorhanden. 
 	 \item \hl{Werden für die Datenabfrage aus \textit{Movebank} weitere Schritte oder Informationen benötigt (Zugangsdaten, vorherige Anfrage bei Provider der Daten, ...) so werden diese vom Nutzer erledigt bzw. bereitgestellt.}
 	 \item Es wird bis zu Milestone 2 ein rudimentäres Vorhersagemodell auf Basis von Positionsdaten aus der Vergangenheit implementiert. Im Laufe der Entwicklung wird in Absprache mit Betreuuern und Kunde ermittelt, welche weiteren Daten und welche weiteren Vorhersagemodelle sinnvoll einbezogen werden können.
\end{itemize} 

% Any other things, that have to be mentioned (e.g. external
% schemata or similar).

\subsection{Einschränkungen}
% Things that could constraint the development
% • Law, Company internals, hardware limitations, safety
% criticality, ...

\hl{todo}

\begin{itemize} 
	\item Die Applikation soll als \textit{Open Source} entwickelt werden. Wird Code oder werden Bibliotheken von Dritten verwendet, so müssen die Lizenzen kompatibel sein.
	\item \hl{Die Abfrage von Daten von Schnittstellen kann limitiert sein.}
	\item Die Internetverbindung des Gerätes kann während der Nutzung für kurze oder längere Zeit ausfallen.
	\item Bei Verwendung des mobilen Datennetzes kann nur eine kleine Übertragungsgeschwindkeit verfügbar sein.
	\item Von \textit{Movebank} bezogene Daten können unvollständig oder fehlerhaft sein.
\end{itemize} 

% The system shall adhere to known standards and built systems and be
% strictly on budget (TODO: Reference to "no fancy new expensive stuff").
% \noindent
% The system shall always enforce the wealth of the animal beings.


%================================================================================================================

\newpage
\section{Mindestanforderungen} \label{mindestanforderungen}

% class diagram, use case diagram, seq diag

\subsection{Funktionen}

\subsubsection{Datenmanagement}

\textbf{R4.)} DatenLadenAusDerMovebank \\
\textbf{Funktion:} Daten sollen aus der Movebank geladen werden. \\
\textbf{Beschreibung:} Das Laden der für die Berechnung benötigten Daten geschieht ausschließlich über die Movebank, die die gegeben Daten zur Verfügung stellt. \\
\textbf{Quelle:} Treffen am 26.04.2018 \\
\textbf{Abhängigkeiten:} - \\

\textbf{R24.)} SpeicherungMovebankNutzerdaten \\
\textbf{Funktion:} Nutzerdaten für die Movebank sollen in den Einstellungen eingetragen werden können \\
\textbf{Beschreibung:} Die App soll in den Einstellungen eine Möglichkeit anbieten, Movebank-Nutzerdaten eintragen zu können, um auch Daten aus der Movebank zu laden, die nur mit einem dortigen Account abrufbar sind. \\
\textbf{Quelle:} Treffen am 26.04.2018 \\
\textbf{Abhängigkeiten:} R4 \\

\textbf{R21.)} OfflineDatenSpeicherung \\
\textbf{Funktion:} Die App soll für die aktuelle Vorhersage Offline-Daten speichern \\
\textbf{Beschreibung:} Die App soll die aktuelle Vorhersage lokal speichern, damit man diese auch ohne aktiven Internetzugang betrachten kann. \\
\textbf{Quelle:} Treffen am 26.04.2018 \\
\textbf{Abhängigkeiten:} R6 \\

\textbf{R22.)} AktualisierungÜberMobileDaten \\
\textbf{Funktion:} Über mobile Daten soll die aktuelle Vorhersage aktualisiert werden können. \\
\textbf{Beschreibung:} Die App soll auch über mobile Daten aktuelle Daten aus der Movebank empfangen und für eine aktuelle Vorhersage verwenden können. \\
\textbf{Quelle:} Treffen am 30.04.2018 \\
\textbf{Abhängigkeiten:} R3, R4, R5, R6, R7, R8, R13 \\

\subsubsection{Vorhersageberechnung}

\textbf{R6.)} BerechnungDesAufenthaltsraumes \\
\textbf{Funktion:} Für einen gegebenen Vogel soll ein wahrscheinlicher Aufenthaltsraum berechnet werden. \\
\textbf{Beschreibung:} Wird ein Vogel ausgewählt, soll aufgrund seiner vergangenen Daten der Aufenthaltsraum berechnet werden, in dem er sich am wahrscheinlichsten befindet. \\
\textbf{Quelle:} Treffen am 26.04.2018 \\
\textbf{Abhängigkeiten:} R4 \\

\textbf{R7.)} AufteilungDerPositionsvorhersage \\
\textbf{Funktion:} Die Positionsvorhersage soll in Zonen mit unterschiedlicher Wahrscheinlichkeit unterteilt werden. \\
\textbf{Beschreibung:} Um die Vorhersage übersichtlichter und verständlicher zu machen, soll die Positionsvorhersage in Zonen eingeteilt werden.  \\
\textbf{Quelle:} Treffen am 30.04.2018 \\
\textbf{Abhängigkeiten:} R6 \\

\textbf{R8.)} VorhersageFürBestimmtenZeitraum \\
\textbf{Funktion:} Die Positionsvorhersage soll für einen gegebenen Zeitraum berechnet werden. \\
\textbf{Beschreibung:} Die Positionsvorhersage soll für einen gegeben Zeitraum ab dem aktuellen Moment berechnet werden \\
\textbf{Quelle:} Treffen am 26.04.2018 \\
\textbf{Abhängigkeiten:} R4 \\

\textbf{R15.)} WählbarerVorhersagezeitraum\\
\textbf{Funktion:} Der Nutzer soll den Zeitraum für die Vorhersage wählen können\\
\textbf{Beschreibung:} Der Nutzer soll die Möglichkeit haben einzustellen für welchen Zeitraum die Vorhersage berechnet wird.
\textbf{Quelle:} Treffen am 26.04.2018\\
\textbf{Abhängigkeiten:} - \\

\textbf{R16.)} WahlDerDatenFürVorhersage\\
\textbf{Funktion:} Der Nutzer soll einstellen können auf welchen Daten die Vorhersage basiert\\
\textbf{Beschreibung:} Der Nutzer soll einstellen können auf welchen der verfügbaren Daten und Parameter die Vorhersage basiert.\\
\textbf{Quelle:} Treffen am 26.04.2018\\
\textbf{Abhängigkeiten:} R4 \\
\\


\subsubsection{Visualisierung}

\textbf{R3.)} VisuelleDarstellungDerVorhersage \\
\textbf{Funktion:} Die Vorhersagen sollen visuell dargestellt werden. \\
\textbf{Beschreibung:} Die Berechnungen zur Vorhersage der wahrscheinlichen Vogelposition sollen visuell in der App angezeigt werden. \\
\textbf{Quelle:} Treffen am 26.04.2018 \\
\textbf{Abhängigkeiten:} - \\


\textbf{R5.)} VisualisierungDerVorhandenenDaten \\
\textbf{Funktion:} Die vorhandenen Daten für einen gegebenen Vogel und Zeitraum sollen visualisiert werden. \\
\textbf{Beschreibung:} Wird ein Vogel ausgewählt, so sollen dessen vorhandenen Daten in einem bestimmtem Zeitraum angezeigt werden können.\\
\textbf{Quelle:} Treffen am 26.04.2018 \\
\textbf{Abhängigkeiten:} R4 \\


\textbf{R12.)} VisualisierungInCesiumBeiWLAN\\
\textbf{Funktion:} Bei WLAN-Zugang sollen die Daten in Cesium visualisiert werden.\\
\textbf{Beschreibung:} Wenn das Smartphone Internetzugang über WLAN hat sollen die getrackten Positionen und und die Vorhersagen mit Cesium visualisiert werden.\\
\textbf{Quelle:} Treffen am 26.04.2018\\
\textbf{Abhängigkeiten:} R4, R5, R6, R7, R8\\


\textbf{R13.)} VisualisierungMitMapbox\\
\textbf{Funktion:} Bei Zugang über das mobile Datennetz sollen die Daten mit Mapbox visualisiert werden.\\
\textbf{Beschreibung:} Wenn das Smartphone Internetzugang ausschließlich über mobieles Internet hat sollen die getrackten Positionen und und die Vorhersagen mit Mapbox visualisiert werden.\\
\textbf{Quelle:} Treffen am 30.04.2018\\
\textbf{Abhängigkeiten:} R4, R5, R6, R7, R8\\


\textbf{R14.)} VisualisierungAufOfflineKarten\\
\textbf{Funktion:} Ohne Internetzugang sollen die Daten auf offline-Karten visualisiert werden.\\
\textbf{Beschreibung:} Falls offline-Karten vorhanden sind sollen diese für die Visualisierung der Vorhersage und der getrackten Positionen verwendet werden. \\
\textbf{Quelle:} Treffen am 30.04.2018\\
\textbf{Abhängigkeiten:} R4, R5, R6, R7, R8\\


\subsubsection{Nutzerinteraktion}

\textbf{R20.)} TouchBasierteBenutzeroberfläche \\
\textbf{Funktion:} Die App soll über eine touch-basierte Benutzeroberfläche verfügen. \\
\textbf{Beschreibung:} Die App soll über eine touch-basierte Benutzeroberfläche bedienbar sein. \\
\textbf{Quelle:} Treffen am 26.04.2018 \\
\textbf{Abhängigkeiten:} - \\

\textbf{R11.)} SucheNachVogel\\
\textbf{Funktion:} In der Software soll nach einem Vogel gesucht werden können\\
\textbf{Beschreibung:} Der Benutzer soll anhand einer ID nach einem bestimmten Vogel suchen können.\\
\textbf{Quelle:} Treffen am 26.04.2018\\
\textbf{Abhängigkeiten:} -\\

\textbf{R18.)} HinweisAufFormalFalscheDaten \\
\textbf{Funktion:} Der Nutzer soll auf einen hohen Anteil formal falscher Daten hingewiesen werden. \\
\textbf{Beschreibung:} Die App soll dem Nutzer einen Hinweis anzeigen, wenn viele formal falsche Daten zur Berechnung verwendet werden mussten, damit dieser einordnen kann wie aussagekräftig das Ergebnis ist. \\
\textbf{Quelle:} Treffen am 30.04.2018 \\
\textbf{Abhängigkeiten:} R4 \\


\subsection{Nicht-funktionale Anforderungen}

\subsubsection{Hardware}

\textbf{R1.)} KorrekteArchitekturDerSmartphones \\
\textbf{Funktion:} Die App sollte auf Smartphones mit Android 5.1 oder höher, Internet-Zugang und einer armeabi, armeabi-v7a oder amr64-v8a Architektur lauffähig sein. \\
\textbf{Beschreibung:} Die App sollte auf Smartphones mit Android 5.1 oder höher, Internet-Zugang und einer armeabi, armeabi-v7a oder amr64-v8a Architektur lauffähig sein. \\
\textbf{Quelle:} Treffen am 30.04.2018 \\
\textbf{Abhängigkeiten:} - \\

\subsubsection{Implementierung}

\textbf{R9.)} ModularitätDerSoftware\\
\textbf{Funktion:} Die Software soll modular gestaltet sein\\
\textbf{Beschreibung:} Die Software soll aus Modulen aufgebaut werden. Diese Module sollen möglichst unabhängig voneinander sein.\\
\textbf{Quelle:} Treffen am 30.04.2018\\
\textbf{Abhängigkeiten:} -\\

\textbf{R10.)} EinheitlicheSchnittstelleFürVorhersagealgorithmen\\
\textbf{Funktion:} Die Software soll über eine einheitliche Schnittstelle für Vorhersagealgorithmen verfügen\\
\textbf{Beschreibung:} Zum einfachen Hinzufügen von weiteren Vorhersagealgorithmen soll die Software über ein einheitliches Interface für diese verfügen\\
\textbf{Quelle:} Treffen am 26.04.2018\\
\textbf{Abhängigkeiten:} R9\\

\subsubsection{Robustheit}

\textbf{R2.)} VerhaltenOhneInternetverbindung \\
\textbf{Funktion:} Die App sollte bei fehlender Internetverbindung mit eingeschränkter Funktionalität weiter nutzbar sein. \\
\textbf{Beschreibung:} Bricht die Verbindung zum Internet ab, sollte die App robust darauf reagieren. Einige Grundfunktionen sollten dabei weiter nutzbar sein. \\
\textbf{Quelle:} Treffen am 30.04.2018 \\
\textbf{Abhängigkeiten:} - \\

\textbf{R17.)} RobustGegenüberFormalFalschenDaten \\
\textbf{Funktion:} Die App soll robust gegenüber formal falschen Daten aus der Movebank sein \\
\textbf{Beschreibung:} Die App soll formal falsche Daten, die sie aus der Movebank bekommt, so aufbereiten, dass das Berechnen und Visualisieren eines Ergebnisses möglich ist. \\
\textbf{Quelle:} Treffen am 26.04.2018 \\
\textbf{Abhängigkeiten:} R4 \\

\subsubsection{Performanz}

\textbf{R19.)} BerechnungsdauerVorhersage \\
\textbf{Funktion:} Der Berechnungsdauer der Vorhersage soll die vom Nutzer erwartete Dauer nicht deutlich überschreiten \\
\textbf{Beschreibung:} Die App soll die Vorhersage in einem adäquaten Zeitraum berechnen und diese dem Nutzer möglichst schnell visualisieren. \\
\textbf{Quelle:} Treffen am 30.04.2018 \\
\textbf{Abhängigkeiten:} R3, R4, R5, R6, R7, R8, R16, R17 \\

\subsubsection{Herstellungsprozess}

\textbf{R23.)} CodeDokumentation \\
\textbf{Funktion:} Programmcode soll kommentiert und dokumentiert werden \\
\textbf{Beschreibung:} Der Programmcode soll kommentiert und dokumentiert werden, um eine effizientere Arbeitsweise im Verlauf des Projektes möglich zu machen. \\
\textbf{Quelle:} Treffen am 26.04.2018 \\
\textbf{Abhängigkeiten:} - \\

 \section{Optionale Anforderungen} \label{optionale_anforderungen}

 Die folgenden Anforderungen sind optional und müssen nicht zwingend erfüllt werden.

\textbf{O1.)} VirtualReality \\
\textbf{Funktion:} Integration von \textit{Virtual Reality}-Modul \\
\textbf{Beschreibung:} Es wird ein zur Verfügung gestelltes Modul in die Applikation integriert, dass die Visualisierungen in einem \textit{Virtual Reality}-Raum visualisiert. \\
\textbf{Quelle:} Treffen am 26.04.2018 \\
\textbf{Abhängigkeiten:} - \\

\textbf{O1.)} ZusätzlicheParameter \\
\textbf{Funktion:} Es werden - neben den Positionsdaten aus der Vergangenheit - noch weitere Parameter für die Berechnung der Vorhersage verwendet.
\textbf{Beschreibung:} Weitere Parameter / Datenquellen fließen in die Berechnung der Vorhersage ein. \\
\textbf{Quelle:} Begleitdokument \\
\textbf{Abhängigkeiten:} - \\

\textbf{O2.)} GewichtungParameter \\
\textbf{Funktion:} Parameter, die in die Berechnung der Vorhersage einfließen sollen gewichtet werden können. \\
\textbf{Beschreibung:} Der Nutzer soll die Möglichkeit haben einzustellen, mit welchem Gewicht die jeweiligen Parameter in die Berechnung der Vorhersage einfließen.\\
\textbf{Quelle:} Begleitdokument \\
\textbf{Abhängigkeiten:} R6 \\

\textbf{O3.)} VariableDarstellung \\
\textbf{Funktion:} Es sind verschiedene Arten der Visualisierung verfügbar.
\textbf{Beschreibung:} Der Nutzer hat die Möglichkeit, das Ergebnis der Vorhersage auf verschiedene Arten zu visualisieren.
\textbf{Quelle:} Begleitdokument \\
\textbf{Abhängigkeiten:} R6, R3 \\

\textbf{O4.)} Evaluation \\
\textbf{Funktion:} Eine Evaluation der Vorhersage ist möglich. \\
\textbf{Beschreibung:} Der Nutzer hat die Möglichkeit, das Ergebnis der Vorhersage zu bewerten. \\
\textbf{Quelle:} Begleitdokument \\
\textbf{Abhängigkeiten:} R6 \\

\textbf{O5.)} MehrereVögel \\
\textbf{Funktion:} Es kann eine Vorhersage für mehrere Vögel gemeinsam getroffen und visualisiert werden. \\
\textbf{Beschreibung:} Für eine gegebene Menge von Vögeln kann eine Vorhersage getroffen und eine Visualisierung erstellt werden. \\
\textbf{Quelle:} Begleitdokument \\
\textbf{Abhängigkeiten:} R6, R3 \\

\textbf{O7.)} Gattung \\
\textbf{Funktion:} Andere Tiere derselben Gattung werden in die Berechnung der Vorhersage einbezogen. \\
\textbf{Beschreibung:} Es werden die Daten von anderen Tieren derselben Gattung für die Berechnung der Vorhersage eines bestimmten Vogels verwendet. \\
\textbf{Abhängigkeiten:} O5, O1 \\

\textbf{O6.)} ÄhnlichkeitenFinden \\
\textbf{Funktion:} Es werden Muster in den Daten aus der Vergangenheit für die Berechnung der Vorhersage verwendet. \\
\textbf{Beschreibung:} Die Analyse von Mustern in Daten aus der Vergangenheit fließt in die Berechnung der Vorhersage ein. \\
\textbf{Quelle:} Begleitdokument \\
\textbf{Abhängigkeiten:} R6, O1 \\





 \section{Nicht-Anforderungen} \label{nicht-anforderungen}

Die App soll keine eigenen Vorhersageverfahren beinhalten, sondern lediglich eine Schnittstelle zur Verfügung stellen, um die vom Kunden zur Verfügung gestellten Modelle/Algorthmen zu integrieren.

\hl{Eine Positionsvorhersage für genaue Zeitpunkte in der Zukunft ist nicht vorgesehen.} Die Vorhersagen beziehen sich auf Zeiträume der nahen Zukunft (Größenordnung von Stunden).


%================================================================================================================

\newpage
\section{Projektzeitplan} \label{zeitplan}

\subsection*{Milestone 1}
Fertigstellung am 15.5.
\begin{itemize} 
	\item \hl{Software-Architektur festgelegt?}
	\item \hl{Software-Design festgelegt?}	
	\item Ein Grundgerüst der Applikation ist fertiggestellt: Es sind folgende Views funktional umgesetzt:
	\begin{itemize} 
		\item Caesium (eine Kartenumgebung wird angezeigt, Navigation möglich)
		\item Mapbox (eine Kartenumgebung wird angezeigt, Navigation möglich)
		\item Diverse UI-Elemente (mindestens Eingabefeld, Button)
	\end{itemize} 
\end{itemize} 

\vspace{1em}

\subsection*{Milestone 2}
Fertigstellung am 5.6.
\begin{itemize} 
	\item Es wird eine Vorhersage mittels eines einfachen Vorhersagemodells berechnet.
	\item Es existiert eine funktionale Visualisierung des Vorhersage-Ergebnisses.
	\item Die Positionsdaten für die Vorhersage werden von der \textit{Movebank}-API bezogen.
	\item Der Aufbau des User-Interfaces ist festgelegt (nicht-funktionale Mockups).
\end{itemize}

\vspace{1em}

\subsection*{Milestone 3}
Fertigstellung am 10.7.
\begin{itemize} 
	\item Anforderungen, die mit eingeschränktem Internetzugriff in Verbindung stehen sind realisiert.
	\item Die momentan angezeigte Vorhersage kann aktualisiert werden (Prüfung, ob neue Daten verfügbar, Neuberechnung; auch unter Verwendung von lediglich mobilem Datennetz)
	\item Die Visualisierung ist vollständig und auf gute Interpretierbarkeit und Ästhetik optimiert.
	\item Das User Interface ist vollständig und auf gute Benutzbarkeit und Ästhetik optimiert.
\end{itemize} 


%================================================================================================================

\newpage
\section{Diagramme} \label{diagramme}

\subsection{Anwendungsfälle}

In folgendem Diagramm sind die wesentlichen Anwendungsfälle der Applikation und ihre Beziehungen dargestellt. Die einzelnen Anwendungsfälle werden im Folgenden detailliert erläutert und eventuell alternative Abläufe beschreiben.

\includegraphics[width = 1\linewidth]{Usecasediagramm.png}

\paragraph{Einloggen} Der Nutzer hat die Möglichkeit, sich bei der \textit{Movebank}-API zu authentifizieren.
\begin{cptenumerate} 
  	 \item Eingabe der Zugangsdaten
  	 \item Übermittlung der Zugangsdaten bei Anfrage an \textit{Movebank}
  	 \begin{cptenumerate}[a.]
  	  	 \item Rückgabe der angefordeten Daten
  	  	 \item Benachrichtigung über fehlgeschlagene Authentifizierung
  	 \end{cptenumerate} 
 \end{cptenumerate}  

 \paragraph{Einstellungen ändern} Der Nutzer hat die Möglichkeit, Einstellungen der Applikation zu ändern.

 \paragraph{Richtung des Vogels anzeigen} Es wird das Ergebnis der Vorhersage visualisiert.

 \paragraph{Vogel auswählen} Der Nutzer kann einen Vogel anhand seiner ID auswählen.
 \begin{cptenumerate} 
     	 \item Eingabe der ID
     	 \item Abfrage der ID von \textit{Movebank}
     	 \begin{cptenumerate}[a.]
     	   	 \item Weiter zur Visualisierung
     	   	 \item Benachrichtung über fehlerhafte oder nicht gefundene ID
     	  \end{cptenumerate}  
    \end{cptenumerate}    

\paragraph{Kartenvisualisierung ansehen} \hl{Der Nutzer betrachtet die Visualisierung des Ergebnisses der Vorhersage auf einer Kartenlandschaft.} 
\begin{cptenumerate}[a.]
 	 \item Bei Internetzugang über WLAN: Darstellung in Cesium.
 	 \item Bei Internetzugang über mobiles Datennetz: Darstellung in Mapbox. 
\end{cptenumerate}


\subsection{Abläufe}

Im folgenden Diagramm wird exemplarisch der Informationsaustausch zwischen beteiligten Komponenten und dessen zeitlicher Ablauf in einer vollständige Sitzung des Nutzers illustriert. 

\includegraphics[width = 1\linewidth]{Sequenzdiagramm.png}

\begin{cptenumerate} 
 	 \item Der Nutzer startet die Applikation.
 	 \item Der Nutzer wählt über das User Interface einen Vogel aus.
 	 \item Die Applikation ruft für den spezifizierten Vogel Daten aus \textit{Movebank} ab.
 	 \item \hl{...}
\end{cptenumerate} 

\end{document}