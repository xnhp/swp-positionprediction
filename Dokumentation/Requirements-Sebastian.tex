\documentclass{article}

\usepackage[utf8]{inputenc}

\setlength{\parindent}{0pt}

\begin{document}

\textbf{R1.)} KorrekteArchitekturDerSmartphones \\
\textbf{Funktion:} Die App sollte auf Smartphones mit Android 5.1 oder höher, Internet-Zugang und einer armeabi, armeabi-v7a oder amr64-v8a Architektur lauffähig sein. \\
\textbf{Beschreibung:} Die App sollte auf Smartphones mit Android 5.1 oder höher, Internet-Zugang und einer armeabi, armeabi-v7a oder amr64-v8a Architektur lauffähig sein. \\
\textbf{Quelle:} Treffen am 30.04.2018 \\
\textbf{Abhängigkeiten:} - \\\\


\textbf{R2.)} VerhaltenOhneInternetverbindung \\
\textbf{Funktion:} Die App sollte bei fehlender Internetverbindung mit eingeschränkter Funktionalität weiter nutzbar sein. \\
\textbf{Beschreibung:} Bricht die Verbindung zum Internet ab, sollte die App robust darauf reagieren. Einige Grundfunktionen sollten dabei weiter nutzbar sein. \\
\textbf{Quelle:} Treffen am 30.04.2018 \\
\textbf{Abhängigkeiten:} - \\\\


\textbf{R3.)} VisuelleDarstellungDerVorhersage \\
\textbf{Funktion:} Die Vorhersagen sollen visuell dargestellt werden. \\
\textbf{Beschreibung:} Die Berechnungen zur Vorhersage der wahrscheinlichen Vogelposition sollen visuell in der App angezeigt werden. \\
\textbf{Quelle:} Treffen am 26.04.2018 \\
\textbf{Abhängigkeiten:} - \\\\


\textbf{R4.)} DatenLadenAusDerMovebank \\
\textbf{Funktion:} Daten sollen aus der Movebank geladen werden. \\
\textbf{Beschreibung:} Das Laden der für die Berechnung benötigten Daten geschieht ausschließlich über die Movebank, die die gegeben Daten zur Verfügung stellt. \\
\textbf{Quelle:} Treffen am 26.04.2018 \\
\textbf{Abhängigkeiten:} - \\\\


\textbf{R5.)} VisualisierungDerVorhandenenDaten \\
\textbf{Funktion:} Die vorhandenen Daten für einen gegebenen Vogel und Zeitraum sollen visualisiert werden. \\
\textbf{Beschreibung:} Wird ein Vogel ausgewählt, sollen dessen vorhandenen Daten in einem bestimmtem Zeitraum angezeigt werden können.\\
\textbf{Quelle:} Treffen am 26.04.2018 \\
\textbf{Abhängigkeiten:} R4 \\\\


\textbf{R6.)} BerechnungDesAufenthaltsraumes \\
\textbf{Funktion:} Für einen gegebenen Vogel soll ein wahrscheinlicher Aufenthaltsraum berechnet werden. \\
\textbf{Beschreibung:} Wird ein Vogel ausgewählt, soll aufgrund seiner vergangenen Daten der Aufenthaltsraum berechnet werden, in der er sich am wahrscheinlichsten befindet. \\
\textbf{Quelle:} Treffen am 26.04.2018 \\
\textbf{Abhängigkeiten:} R4 \\\\



\textbf{R7.)} AufteilungDerPositionsvorhersage \\
\textbf{Funktion:} Die Positionsvorhersage soll in Zonen mit unterschiedlicher Wahrscheinlichkeit unterteilt werden. \\
\textbf{Beschreibung:} Um die Vorhersage übersichtlichter und verständlicher zu machen, soll die Positionsvorhersagen in Zonen eingeteilt werden.  \\
\textbf{Quelle:} Treffen am 30.04.2018 \\
\textbf{Abhängigkeiten:} R6 \\\\


\textbf{R8.)} VorhersageFürBestimmtenZeitraum \\
\textbf{Funktion:} Die Positionsvorhersage soll für einen gegebenen Zeitraum berechnet werden. \\
\textbf{Beschreibung:} Die Positionsvorhersage soll für einen gegeben Zeitraum ab dem aktuellen Moment berechnet werden \\
\textbf{Quelle:} Treffen am 26.04.2018 \\
\textbf{Abhängigkeiten:} R4 \\\\


\end{document}